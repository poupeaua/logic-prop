
\documentclass{article}
\usepackage{graphicx}
\usepackage[utf8]{inputenc}
\usepackage[english]{babel}
\usepackage{amsmath, amsthm, amssymb}
\usepackage{stmaryrd}

\newtheorem{prop}{Property}

\begin{document}

\title{Introduction to \LaTeX{}}
\author{Alexandre Poupeau}

\maketitle


\section*{Exercice 1}

For the rest of the exercice 1, we assume that : \\

$ p\in CNF \iff \exists k \in \mathbb{N}$ and $\forall i \in \llbracket 1, k\rrbracket,  \exists J_i \in \mathbb{N}$ such that $p = \displaystyle\bigwedge_{1 \leq i \leq k} C_i = \displaystyle\bigwedge_{1 \leq i \leq k} (\displaystyle\bigvee_{1 \leq j  \leq J_i} l_{i,j})$ where $l_{i,j}$ is a literal. $C_i$ is the representation we will used to describe "clausulas".

\paragraph{1.1) Prove that all logical propositional formulas puden escribirse en $CNF$.}

In order to demonstrate this we will prove two properties first.

\begin{prop}
  \label{prop1.1}
  $P_n :$ If $(\Phi,\Omega)\in CNF$ and are especially of the following form : $\Phi = \displaystyle\bigwedge_{1 \leq i \leq k} C_i$ and $\Omega = \displaystyle\bigwedge_{1 \leq i \leq k^{\prime}} C^{\prime}_i$ with $k,k^{\prime} \leq n  \implies (\Phi \bigvee \Omega)\in CNF$
\end{prop}

\begin{proof}

  Case $n = 1$ :
  \begin{equation*}
    \begin{split}
      \Phi \displaystyle\bigvee \Omega & = \displaystyle\bigvee_{1 \leq j  \leq J^{\Phi}_1} l^{\Phi}_{1,j} \vee \displaystyle\bigvee_{1 \leq j  \leq J^{\Omega}_1} l^{\Omega}_{1,j}\\
       & = \displaystyle\bigwedge_{1 \leq i \leq k} (\displaystyle\bigvee_{1 \leq j  \leq J^{\prime}_i} l^{\prime}_{i,j}) \\
    \end{split}
  \end{equation*}

  Where $k=1$, $J^{\prime}_1 = J^{\Phi}_1 + J^{\Omega}_1$ and $l^{\prime}=\left\{
                \begin{array}{ll}
                  l^{\Phi}_{1,j}$ , $\forall j \in \llbracket 1, J^{\Phi}_1\rrbracket\\
                  l^{\Omega}_{1,j}$, $\forall j \in \llbracket J^{\Phi}_1 J^{\Phi}_1+J^{\Omega}_1\rrbracket
                \end{array}
              \right.$

\end{proof}

\nocite{*}

\section*{Exercice 2}

\paragraph{1.1) Prove that all natural number can be expressed in factorial representation.}

First, we need the show this property :

\begin{prop}
  \label{prop1}
  $P_n : \displaystyle\sum_{k=0}^n (k*k!) +1 = (n+1)!$
\end{prop}

We can prove this quite easily using induction.

\begin{proof}

  Case $n=0$ : $0*0! + 1 = (0+1)! = 1$\\

  Let us suppose that $P_n$ is true. We have to prove that $P_{n+1}$ is true.

  \begin{equation*}
    \begin{split}
      \displaystyle\sum_{k=0}^{n+1} (k*k!) +1 & = (n+1)*(n+1)! + \displaystyle\sum_{k=0}^n (k*k!) +1 \\
       & = (n+1)*(n+1)! + (n+1)! \\
       & = (n+2)!
    \end{split}
  \end{equation*}

  Thus the property is true $\forall n \geq  0$.
\end{proof}

Now we are going to prove the following property which is the one we want to prove here.

\begin{prop}
  $P_n : $ The natural number $n$ can be written using factorial representation. This means that $\exists k \in \mathbb{N}$ and $\forall i \in \llbracket 0, k\rrbracket$, $ \exists a_i \in \llbracket 0, i\rrbracket$ such that $n = \displaystyle\sum_{i=0}^k a_i*i!$. The factorial representation of $n$ is represented by $a_k...a_1a_0$ where $k$ is the smallest natural number that satisfies the previous result.
\end{prop}

\begin{proof}

  Case $n=0$ : $0 = a_0*0! = \displaystyle\sum_{i=0}^k a_i*i!$ with $k=0$ and $a_0 = 0$.\\

  Let us suppose that $P_n$ is true. We have to prove that $P_{n+1}$ is true. \\

  As this is our induction hypothesis, we know that $\exists k \in \mathbb{N}$ and $\forall i \in \llbracket 0, k\rrbracket$, $ \exists a_i \in \llbracket 0, i\rrbracket$ such that $n = \displaystyle\sum_{i=0}^k a_i*i!$. For convenience, let us take the smallest $k$ such that $\forall i \in \llbracket 0, k\rrbracket$, $ \exists a_i \in \llbracket 0, i\rrbracket$ such that $n = \displaystyle\sum_{i=0}^k a_i*i!$. This way we do not work with useless zeros.\\

  Let $i^* = min(i \in \llbracket 0, k+1\rrbracket \mid a_i \ne i)$.\\

  Here is a property we want to prove :

  \begin{itemize}
    \item If $i^* \leq k$, then $n+1 = \displaystyle\sum_{i=0}^{k^\prime} a_i^\prime*i!$ where $k^\prime = k$ and $a_i^\prime=\left\{
                  \begin{array}{ll}
                    0 $, $\forall i < i^*\\
                    a_i $, $\forall i > i^*\\
                    a_i +1 $, if $i=i^*
                  \end{array}
                \right.$
    \item If $i^* = k+1$, then $n+1 = 1*(k+1)! = \displaystyle\sum_{i=0}^{k^\prime} a_i^\prime*i!$ where $k^\prime = k+1$ and $a_i^\prime=\left\{
                  \begin{array}{ll}
                    0 $, $\forall i < i^*\\
                    1 $, if $i=i^*
                  \end{array}
                \right.$
  \end{itemize}

  Proving this previous property implies that $n+1$ can be written in factorial representation. \\

  Let us consider the first case where $i^* \leq k$. Then $\forall i \in \llbracket 0, i^*-1\rrbracket$, $a_i = i$ and $a_{i^*} \in \llbracket 0, i^*-1\rrbracket$.

  \begin{equation*}
    \begin{split}
      n+1 & = \displaystyle\sum_{i=0}^k a_i*i! +1 \\
       & = \displaystyle\sum_{i=i^*+1}^k a_i*i! + a_{i^*}*i^*! + \displaystyle\sum_{i=0}^{i^*-1} (i*i!) +1 \\
       & = \displaystyle\sum_{i=i^*+1}^k a_i*i! + a_{i^*}*i^*! + i^*!  \hspace{1cm} \text{Because of the property 1} \\
       & = \displaystyle\sum_{i=i^*+1}^k a_i*i! + (a_{i^*}+1)*i^*! \\
       & = \displaystyle\sum_{i=0}^{k^\prime} a_i^\prime*i!
    \end{split}
  \end{equation*}

  Where $k^\prime = k$ and $a_i^\prime=\left\{
                \begin{array}{ll}
                  0 $, $\forall i < i^*\\
                  a_i $, $\forall i > i^*\\
                  a_i +1 $, if $i=i^*
                \end{array}
              \right.$ \\

  Let us now consider the second case where $i^* = k+1$, which is a special case. Here $\forall i \in \llbracket 0, k\rrbracket$, $a_i = i$.

  \begin{equation*}
    \begin{split}
      n+1 & = \displaystyle\sum_{i=0}^k i*i! +1 \\
       & = (k+1)! \hspace{1cm} \text{Because of the property 1} \\
       & = \displaystyle\sum_{i=0}^{k^\prime} a_i^\prime*i!
    \end{split}
  \end{equation*}

  Where $k^\prime = k+1$ and $a_i^\prime=\left\{
                \begin{array}{ll}
                  0 $, $\forall i < i^*\\
                  1 $, if $i=i^*
                \end{array}
              \right.$ \\

  In both cases, $n+1$ can be expressed in factorial representation. \\

  To conclude, we have seen that the property is true when $n=0$, we have proven that $P_n$ implies that $P_{n+1}$ is true. Thus the property stands $\forall n \geq  0$.

\end{proof}

\paragraph{1.2) Prove that this expression is unique (there is not two ways to express a natural number in factorial representation).}

\begin{proof}

Let us suppose there exists $n$, a natural number such that $n = \displaystyle\sum_{i=0}^k a_i*i!$ and $n = \displaystyle\sum_{i=0}^{k^\prime} b_i*i!$ where there is at least a $i$ such that $a_i \ne b_i$. \\

Without loss of generality, we can write that $k^\prime = k$ (because in the case of $k^\prime > k$ we can say that all the $a_i=0$, $ \forall i \in \llbracket k+1, k^\prime\rrbracket$). \\

Let $l=max(i \in \llbracket 0, k\rrbracket \mid a_i \ne b_i)$, $A=(a_0, a_1, \dots , a_{l-1})$ and $B=(b_0, b_1, \dots , b_{l-1})$. We define $\phi_{A, l}$ as follows : $\phi_{A, l} = \dfrac{1}{l!}\displaystyle\sum_{i=0}^{l-1} a_i*i!$\\

Hence we have that :

\begin{alignat*}{3}
  \displaystyle\sum_{i=0}^k a_i*i! &= \displaystyle\sum_{i=0}^{k} b_i*i! & &\Rightarrow {} & a_l*l! + \displaystyle\sum_{i=0}^{l-1} a_i*i! &= b_l*l! + \displaystyle\sum_{i=0}^{l-1} b_i*i! \\
  & & &\Rightarrow  & a_l + (\dfrac{1}{l!}\displaystyle\sum_{i=0}^{l-1} a_i*i!) &= b_l + (\dfrac{1}{l!}\displaystyle\sum_{i=0}^{l-1} b_i*i!) \\
  & & &\Rightarrow  & a_l + \phi_{A, l} &= b_l + \phi_{B, l} \\
  & & &\Rightarrow  & (a_l-b_l) &= (\phi_{B, l}-\phi_{A, l})
\end{alignat*}

By definition of $a_l$ and $b_l$, we know that $(a_l-b_l) \in \mathbb{Z}\backslash \{0\}$. Moreover, by definition, we know that $\forall i \in \llbracket 0, l-1\rrbracket$, $a_i \leq i$ and $b_i \leq i$. Thus :\\

Thus $0 \leq \phi_{A, l} \leq \dfrac{1}{l!}\displaystyle\sum_{i=0}^{l-1} i*i! \leq \dfrac{1}{l!}(\displaystyle\sum_{i=0}^{l-1} (i*i!) +1 -1) = \dfrac{l!-1}{l!} = 1-\dfrac{1}{l!} < 1$. Then $-1 > -\phi_{A, l} \geq 0$. We can deduce the same results for $\phi_{B, l}$, and especially $0 \leq \phi_{B, l} < 1$. Given those two results, we have that :

\begin{equation*}
  -1 < \phi_{B, l}-\phi_{A, l} < 1
\end{equation*}

This result is absurd since we have that $(a_l-b_l) = (\phi_{B, l}-\phi_{A, l})$ and that $(a_l-b_l) \in \mathbb{Z}\backslash \{0\}$. \\

Hence we showed that supposing there exists a natural number that can be expressed in two different ways in factorial representation implies a absurd result. Therefore the factorial representation a all natural numbers is different.

\end{proof}

\section*{Exercice 3}

\paragraph{1.1) Prove that all natural number can be expressed in Fibonacci representation.}

\begin{prop}
  $P_n : $ The natural number $n$ can be written using Fibonacci representation. This means that $\exists k \in \mathbb{N}$ and $\forall i \in \llbracket 0, k\rrbracket$, $ \exists a_i \in \llbracket 0, 1\rrbracket$ such that $n = \displaystyle\sum_{i=0}^k a_i*f_{i+2}$ and $\forall i \in \llbracket 0, k-1\rrbracket$, $a_ia_{i+1} = 0$. The Fibonacci representation of $n$ is represented by $a_k...a_1a_0$ where $k$ is the smallest natural number that satisfies the previous result.
\end{prop}

The very first thing we have to consider is that the Fibonacci number associated to $a_i$ is $f_{i+2}$, not $f_i$. The other important thing, is the constraint $\forall i \in \llbracket 0, k-1\rrbracket$, $a_ia_{i+1} = 0$. This implies that we can not have two consecutive $a_i$ equal to one. Now that we have a better understanding of $P_n$, we are going to prove it.\\

\begin{proof}
  Case $n=0$ : $0 = 0 * 1 = \displaystyle\sum_{i=0}^k a_i*f_{i+2}$ where $k=0$ and $a_0 = 0$. \\

  Let us suppose $P_n$ is true. We want to prove that $P_{n+1}$ is true. As this is our induction hypothesis, we know that $\exists k \in \mathbb{N}$ and $\forall i \in \llbracket 0, k\rrbracket$, $ \exists a_i \in \llbracket 0, 1\rrbracket$ such that $n = \displaystyle\sum_{i=0}^k a_i*f_{i+2}$ and $\forall i \in \llbracket 0, k-1\rrbracket$, $a_ia_{i+1} = 0$. For convenience, let us take the smallest $k$ such that $\forall i \in \llbracket 0, k\rrbracket$, $ \exists a_i \in \llbracket 0, 1\rrbracket$ such that $n = \displaystyle\sum_{i=0}^k a_i*f_{i+2}$ and $\forall i \in \llbracket 0, k-1\rrbracket$, $a_ia_{i+1} = 0$. This way we do not work with useless zeros. We have $a_k = 1$ in all cases except if $n=0$.\\

  Let $i^*=min(i \in \llbracket 0, k+1\rrbracket \mid a_i + a_{i+1} = 0)$. By this definition, we have that $a_{i^*}=0$ and $a_{i^*+1}=0$.\\

  Here is a property we want to prove :

  \begin{itemize}
    \item If $i^* = 0$, then $n+1 = \displaystyle\sum_{i=0}^{k^\prime} a_i^\prime*f_{i+2}$ where $k^\prime = k = 0$ and $a_0^\prime=1$.
    \item If $0< i^* \leq k$, then $n+1 = \displaystyle\sum_{i=0}^{k^\prime} a_i^\prime*f_{i+2}$ where $k^\prime = k$ and $a_i^\prime=\left\{
                  \begin{array}{ll}
                    0 $, $\forall i < i^*\\
                    a_i $, $\forall i > i^*\\
                    1 $, if $i=i^*
                  \end{array}
                \right.$
    \item If $i^* = k+1$, then $n+1 = \displaystyle\sum_{i=0}^{k^\prime} a_i^\prime*f_{i+2}$ where $k^\prime = k+1$ and $a_i^\prime=\left\{
                  \begin{array}{ll}
                    0 $, $\forall i < i^*\\
                    1 $, if $i=i^*
                  \end{array}
                \right.$
    \end{itemize}

    Moreover the property states that in all cases $\forall i \in \llbracket 0, k^\prime-1\rrbracket$, $a_i^\prime a_{i+1}^\prime = 0$.\\

    Proving the previous property implies that $n+1$ can be written in Fibonacci representation. So let us prove it now : \\

    The case $i^* = 0$ is trivial. This implies that $n=0$, thus $n+1= 1*1 = 1*f_2$. We obviously have $\forall i \in \llbracket 0, k^\prime-1\rrbracket$, $a_i^\prime a_{i+1}^\prime = 0$.\\

    In the case $0 < i^* \leq k$, we can say that
    $\left\{\begin{array}{ll}
        \forall i \in \llbracket 0, i^*-1\rrbracket$, $a_ia_{i+1} = 0\\
        \forall i \in \llbracket 0, i^*-1\rrbracket$, $a_i + a_{i+1} \ne 0\\
        a_{i^*-1} = 1
      \end{array}\right.$ We can notice that $i^*$ is necessarily different from $k$ or $k-1$.\\

    In this case, $n$ can be written in this way $n = \displaystyle\sum_{i=i^*+2}^k a_i*f_{i+2} + \displaystyle\sum_{i=0}^{i^*-1} a_i*f_{i+2}$.
    Here we can divide the problem in two parts : $i^*-1$ is even and $i^*-1$ is odd. \\

    If $i^*-1$ is even, $\forall i \in \mathbb{N}$ such that $2i \leq i^*-1$, $a_{2i} = 1$ y $a_{2i-1} = 0$. Thus $n = \displaystyle\sum_{i=i^*+2}^k a_i*f_{i+2} + \displaystyle\sum_{i=0}^{(i^*-1)/2} f_{2i+2}$. As $f_1 = 1$,
    and by the property of the Fibonacci sequence $f_{n+2} = f_{n+1} + f_n$, we can show that :

    \begin{equation*}
      \begin{split}
        n+1 & = \displaystyle\sum_{i=i^*+2}^k a_i*f_{i+2} + \displaystyle\sum_{i=0}^{(i^*-1)/2} f_{2i+2} + 1 \\
         & = \displaystyle\sum_{i=i^*+2}^k a_i*f_{i+2} + \underbrace{\displaystyle\sum_{i=0}^{(i^*-1)/2} f_{2i+2} + f_1}_{= f_{i^*+2}} \hspace{1cm} \text{Property of the Fibonacci sequence} \\
         & = \displaystyle\sum_{i=0}^{k^\prime} a_i^\prime*f_{i+2}
      \end{split}
    \end{equation*}

    If $i^*-1$ is odd, $\forall i \in \mathbb{N}$ such that $2i+1 \leq i^*-1$, $a_{2i} = 0$ y $a_{2i+1} = 1$. Thus $n = \displaystyle\sum_{i=i^*+2}^k a_i*f_{i+2} + \displaystyle\sum_{i=0}^{(i^*-2)/2} f_{2i+3}$. As $f_2 = 1$,
    and by the property of the Fibonacci sequence $f_{n+2} = f_{n+1} + f_n$, we can show that :

    \begin{equation*}
      \begin{split}
        n+1 & = \displaystyle\sum_{i=i^*+2}^k a_i*f_{i+2} + \displaystyle\sum_{i=0}^{(i^*-2)/2} f_{2i+3} + 1 \\
         & = \displaystyle\sum_{i=i^*+2}^k a_i*f_{i+2} + \underbrace{\displaystyle\sum_{i=0}^{(i^*-2)/2} f_{2i+3} + f_2}_{= f_{i^*+2}} \hspace{1cm} \text{Property of the Fibonacci sequence} \\
         & = \displaystyle\sum_{i=0}^{k^\prime} a_i^\prime*f_{i+2}
      \end{split}
    \end{equation*}

    In both cases ($i^*-1$ even and odd), $k^\prime = k$ and $a_i^\prime=\left\{
                  \begin{array}{ll}
                    0 $, $\forall i < i^*\\
                    a_i $, $\forall i > i^*\\
                    1 $, if $i=i^*
                  \end{array}
                \right.$ \\

    In both cases $\forall i > i^* $, $a_i^\prime=a_i$ so as we have that $\forall i \in \llbracket i^*+1, k-1\rrbracket$, $a_ia_{i+1} = 0$ we have that $\forall i \in \llbracket i^*+1, k^\prime-1\rrbracket$, $a_i^\prime a_{i+1}^\prime = 0$. Moreover, $\forall i < i^* $, $a_i^\prime=0$ so $\forall i \in \llbracket 0, i^*-1\rrbracket$, $a_i^\prime a_{i+1}^\prime = 0$. Finally, we know that $a_{i^*}^\prime a_{i^*+1}^\prime = 0$ because $a_{i^*+1}^\prime = a_{i^*+1} = 0$ by definition of $i^*$. So $\forall i \in \llbracket 0, k^\prime-1\rrbracket$, $a_i^\prime a_{i+1}^\prime = 0$\\

    If $i^* = k+1$, then if $k$ is even $n+1 = \displaystyle\sum_{i=0}^{k/2} f_{2i+2} +1 = \displaystyle\sum_{i=0}^{k/2} f_{2i+2} + f_1 = f_{k+3}$. If $k$ is odd $n+1 = \displaystyle\sum_{i=0}^{(k-1)/2} f_{2i+3} +1 = \displaystyle\sum_{i=0}^{(k-1)/2} f_{2i+3} + f_2 = f_{k+3}$. \\

    So in both cases ($k$ even and odd), $n+1 = \displaystyle\sum_{i=0}^{k^\prime} f_{i+2}$ where $k^\prime = k+1$ and $a_i^\prime=\left\{
                  \begin{array}{ll}
                    0 $, $\forall i < i^*\\
                    1 $, if $i=i^*
                  \end{array}
                \right.$

    Then obviously, as only $a_{i^*} = 0$, $\forall i \in \llbracket 0, k^\prime-1\rrbracket$, $a_i^\prime a_{i+1}^\prime = 0$. \\

    To conclude, we have seen that the property is true when $n=0$, we have proven that $P_n$ implies that $P_{n+1}$ is true. Thus the property stands $\forall n \geq  0$.

\end{proof}

\paragraph{1.2) Prove that this expression is unique (there is not two ways to express a natural number in Fibonacci representation).}

In order to proove this, we are going to use the same idea or method from the previous part.

\begin{proof}
  Let us suppose there exists a natural number $n$ such that $n = \displaystyle\sum_{i=0}^k a_if_{i+2} = \displaystyle\sum_{i=0}^{k^\prime} b_if_{i+2}$ (like the previous exercice we can assume without loss of generality that $k^\prime = k$) where $\forall i \in \llbracket 0, k\rrbracket$, $a_i \in \llbracket 0, 1\rrbracket$ and $b_i \in \llbracket 0, 1\rrbracket$, where there exists at least a $i \in \llbracket 0, k-1\rrbracket$ such that $a_i \ne b_i$. We have that $\forall i \in \llbracket 0, k-1\rrbracket$, $a_ia_{i+1} = 0$ and $\forall i \in \llbracket 0, k-1\rrbracket$, $b_ib_{i+1} = 0$. \\

  Let $l=max(i \in \llbracket 0, k\rrbracket \mid a_i \ne b_i)$. We know that $|a_l-b_l| = 1$. Let $A=(a_0, a_1, \dots , a_{l-1})$ and $B=(b_0, b_1, \dots , b_{l-1})$. We define $\beta_{A, l}$ as follows : $\beta_{A, l} = \dfrac{1}{f_{l+2}}\displaystyle\sum_{i=0}^{l-1} a_if_{i+2}$.\\

  \begin{alignat*}{3}
    \displaystyle\sum_{i=0}^k a_if_{i+2} &= \displaystyle\sum_{i=0}^{l-1} b_if_{i+2} & &\Rightarrow {} & a_lf_{l+2} + \displaystyle\sum_{i=0}^k a_if_{i+2} &= b_lf_{l+2} + \displaystyle\sum_{i=0}^{l-1} b_if_{i+2} \\
    & & &\Rightarrow  & a_l + (\dfrac{1}{f_{l+2}}\displaystyle\sum_{i=0}^{l-1} a_if_{i+2}) &= b_l + (\dfrac{1}{f_{l+2}}\displaystyle\sum_{i=0}^{l-1} b_if_{i+2}) \\
    & & &\Rightarrow  & a_l + \beta_{A, l} &= b_l + \beta_{B, l} \\
    & & &\Rightarrow  & (a_l-b_l) &= (\beta_{B, l}-\beta_{A, l})
  \end{alignat*}

 Because of the constraint that $\forall i \in \llbracket 0, k-1\rrbracket$, $a_ia_{i+1} = 0$, if $l-1$ is even :

 $$0 \leq \beta_{A, l} = \dfrac{1}{f_{l+2}}(\displaystyle\sum_{i=0}^{l-1} a_if_{i+2}) = \dfrac{1}{f_{l+2}}(f_{l+1} + f_{l-1} + \dots + f_4 + f_2 + f_1 -1) = \dfrac{1}{f_{l+2}}(f_{l+2} -1) = 1-\dfrac{1}{f_{l+2}} < 1$$

 If $l-1$ is odd, we obtain the same result :

 $$0 \leq \beta_{A, l} = \dfrac{1}{f_{l+2}}(\displaystyle\sum_{i=0}^{l-1} a_if_{i+2}) = \dfrac{1}{f_{l+2}}(f_{l+1} + f_{l-1} + \dots + f_5 + f_3 + f_2 -1) = \dfrac{1}{f_{l+2}}(f_{l+2} -1) = 1-\dfrac{1}{f_{l+2}} < 1$$

 Then $-1 > -\beta_{A, l} \geq 0$. We can deduce the same results for $\beta_{B, l}$, and especially $0 \leq \beta_{B, l} < 1$. Given those two results, we have that :

\begin{equation*}
  -1 < \beta_{B, l}-\beta_{A, l} < 1
\end{equation*}

Finally, we have that $1 = |a_l-b_l| = \beta_{B, l}-\beta_{A, l} < 1$ which is absurd. Thus the Fibonacci representation is unique for all $n \in \mathbb{N}$.

\end{proof}

\appendix
\bibliographystyle{ieeetr}
\bibliography{demonstrations}

\end{document}
